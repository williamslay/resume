% !TEX program = xelatex

\documentclass{resume}
%\usepackage{zh_CN-Adobefonts_external} % Simplified Chinese Support using external fonts (./fonts/zh_CN-Adobe/)
\usepackage{zh_CN-Adobefonts_internal} % Simplified Chinese Support using system fonts

\begin{document}
\pagenumbering{gobble} % suppress displaying page number

\name{Lei Shangyuan}

\basicInfo
{
\href{https://github.com/williamslay}{\raisebox{-0.05\height}\faGithub\ williamslay} \ $|$ \
%\href{https://mysite.com}{\raisebox{-0.05\height}\faGlobe \ mysite.com} \ $|$ \
\href{mailto:williamslay.lsy@outlook.com}{\raisebox{-0.05\height}\faEnvelope \ williamslay.lsy@outlook.com} \ $|$ \
\href{tel:18851986021}{\raisebox{-0.05\height}\faMobile 18851986021} \\
}

\section{\faGraduationCap\ Education}
\datedsubsection{\textbf{Xidian University}, Shaanxi, China}{2009 -- 2013}
\textit{B.S.} in Electronics Engineering (EE)

\section{\faUsers\ Experience}
\datedsubsection{\textbf{FLAG Inc.} California, America}{2012 -- Present}
\role{Summer Intern}{Manager: xxx}
Brief introduction: xxx.
\begin{itemize}
  \item Implemented xxx feature
  \item Optimized xxx 5\%
  \item xxx
\end{itemize}

\datedsubsection{\textbf{xxx Projects}}{Jan. 2015 -- Present}
\role{C, Python, Django, Linux}{Individual Projects, collaborated with xxx}
Brief introduction: xxx
\begin{itemize}
  \item Implemented xxx feature
  \item Optimized xxx 5\%
  \item xxx
\end{itemize}

\datedsubsection{\textbf{\LaTeX\ résumé template}}{May. 2015 -- Present}
\role{\LaTeX, Maintainer}{Individual Projects}
An elegant \LaTeX\ résumé template, https://github.com/billryan/resume
\begin{itemize}
  \item Easy to be further customized or extended
  \item Full support for unicode characters (e.g. CJK) with \XeLaTeX\
  \item FontAwesome 4.5.0 support
\end{itemize}

% Reference Test
%\datedsubsection{\textbf{Paper Title\cite{zaharia2012resilient}}}{May. 2015}
%An xxx optimized for xxx\cite{verma2015large}
%\begin{itemize}
%  \item main contribution
%\end{itemize}

\section{\faCogs\ Skills}
\begin{itemize}[parsep=0.5ex]
  \item Programming Languages: C == Python > C++ > Java
  \item Platform: Linux
  \item Development: Web, xxx
\end{itemize}

\section{\faHeartO\ Honors and Awards}
\datedline{\textit{\nth{1} Prize}, Award on xxx }{Jun. 2013}
\datedline{Other awards}{2015}

\section{\faInfo\ Miscellaneous}
\begin{itemize}[parsep=0.5ex]
  \item Blog: http://your.blog.me
  \item GitHub: https://github.com/username
  \item Languages: English - Fluent, Mandarin - Native speaker
\end{itemize}

\clearpage
% the chinese resume
\name{雷尚远}

\basicInfo
{
\href{https://github.com/williamslay}{\raisebox{-0.05\height}\faGithub\ williamslay} \ $|$ \
%\href{https://mysite.com}{\raisebox{-0.05\height}\faGlobe \ mysite.com} \ $|$ \
\href{mailto:williamslay.lsy@outlook.com}{\raisebox{-0.05\height}\faEnvelope \ williamslay.lsy@outlook.com} \ $|$ \
\href{tel:18851986021}{\raisebox{-0.05\height}\faMobile 18851986021} \\
}
\section{\faGraduationCap\ 教育经历}
\datedsubsection{\textbf{南京邮电大学}, \textit{计算机科学与技术} }{2019 -- 2023}
在校成绩:3.56/5.0,专业排名 86/256(33.59\%)

\section{\faUsers\ 工作经历}
\datedsubsection{\textbf{外企德科-华为OD研发工程师项目} }{2024.04 -- 2025.09}
\role{系统软件开发工程师(D2)}{数据通信产品线,网关软转发项目组}
所在项目组为安全网关,园区网络等产业交付高性能转发(High-Performance Forwarding, HPF)组件。项目组主要负责交换机,防火墙等产品的三层数据面软件转发业务。
\begin{itemize}
  \item 负责日常基础转发机制问题的接口定位,问题单解决,以及承担项目组每月业务迭代需求交付。主要负责过的模块包括:基础DFX机制(包括报文流/包统计,基础示踪模块),隧道转发业务(包括SASE业务,vxlan业务等),转发管理配置模块等。
  \item 组件小型化问题专项。负责分析与解决组件产出二进制文件体积小型化,内存占用体积优化以及CPU热点函数性能优化等小型化专项问题,优化交付件编译工程产出。
  \item 担任组内的持续集成构建工程师(轮流)。负责保障组内日常需求开发代码持续推送集成构建主干,维护本地仓库主干,保障联调需求协调,解决组内编译工具链及平台相关问题。
\end{itemize}

\datedsubsection{\textbf{武汉拓森信息科技有限公司}}{2021.07 -- 2021.10}
\role{Web前端开发实习生}{}
采用传统前后端分离开发方式,负责编写华中科技大学教务管理系统模块。负责设计相关子模块页面原型,以及利用Vue组件实现前端业务页面及功能逻辑,负责设计了部分后端数据库表。

% Reference Test
%\datedsubsection{\textbf{Paper Title\cite{zaharia2012resilient}}}{May. 2015}
%An xxx optimized for xxx\cite{verma2015large}
%\begin{itemize}
%  \item main contribution
%\end{itemize}

\section{\faCogs\ 个人项目经历}
\datedsubsection{\href{https://github.com/williamslay/NJUPT_Bachelor}{\textbf{毕业设计}:定向覆盖模糊测试工具的设计与实现}}{2023.2 -- 2023.6}
基于现有的模糊测试工具框架AFL以及AFLGo进行定向覆盖测试策略的设计与集成,并针对相应的公开通用漏洞集(CVE)复现及定向实验对比测试。主要工作量:改写AFLGo中llvm插桩计算过程,精确了程序控制流图的距离计算指标;改写AFLGo插桩过程,增加定向覆盖适应度指标。

\datedsubsection{\href{https://github.com/williamslay/NJUICSPA_2023}{\textbf{兴趣项目}:南京大学2023秋季PA}}{2024.04 -- 至今}
工作结余时间独立完成PA0-PA2部分,包括所有选做内容。选择riscV-32指令集版本,当前已正确实现NEMU模拟器,通过所有相关测试集。选做内容完成至对声卡外设的支持,可正确播放bad apple以及运行红白机模拟器中超级马里奥游戏。

\datedsubsection{\href{https://github.com/williamslay/win32_clock}{\textbf{课程设计项目}:基于win32汇编设计实现的时钟程序}}{2022.05}
为期两周,主要是利用win32汇编语言设计实现一个时钟程序。总代码量约一千行。

\section{\faInfo\ 技能与获奖荣誉}
\begin{itemize}[parsep=0.5ex]
  \item CET6: 540
  \item 会使用linux基本指令以及工具(readelf,objdump等),对git工具比较熟悉;在工作中会借助ChatGpt等AI工具辅助完成简单shell脚本工具,辅助完成简单的任务自动化流程;主要熟悉C以及C++开发工具链
  \item 获奖荣誉:国家励志奖学金、江苏省十一届“三创赛”省一等奖、华为可信能力认证考试工作级
\end{itemize}

%% Reference
%\newpage
%\bibliographystyle{IEEETran}
%\bibliography{mycite}
\end{document}
